\documentclass{article}
\begin{document}

\title{N-Body simulations of star-star encounters}
\author{Vatsal Mandalia}
\maketitle

\begin{abstract}
{\normalsize{N-Body simulations are carried out on the Solar system using the predictor-corrector method. }}

\end{abstract}

\section{\textbf{Introduction}}
{\normalsize{Stars like our Sun are born in groups/clusters with other stars. All of them having a similar age, only a difference in their masses. Gravitational interactions in the clusters cause changes in the dynamical properties of the members. The evolution of this group of stars can be produced in N-Body simulations. In this project, N-Body simulations are carried out on the Solar System. Different methods were used which gave varying accuracies. 

Star clusters are important targets in astronomical observations. A observation of the cluster represents a snapshot in time. Detailed spectroscopic observations can help in producing an H-R diagram for the cluster. This plot provides important information about the cluster like the different populations of stars, the age of the cluster etc. However, to look at how the cluster will vary with time under the influence of gravity, theoretical simulations represent the perfect choice. Here, by knowing the initial values of the of the dynamical properties of the members of the Solar system, we perform N-Body simulations.

Two methods were used in this project over the two semesters. In the first semester, a second order Euler's method was employed. At the base of these methods lies two first order differential equations for acceleration and velocity. The Euler's method obtains the future position/velocity as an approximate solution of the differential equations. The errors occurring from this method go as $dt$$^{2}$. Higher accuracy is obtained with a smaller timestep, with a consequence of long computational time. Therefore, a higher order method is used to give a higher accuracy. A fourth order predictor-corrector method is applied in this semester for the N-Body simulations.}}

\section{Method}
{\normalsize{
The code for the fourth order predictor corrector method is divided into blocks for ease in error checking. Newton's law of gravity is the main theory behind these simulations of the Solar System. The second order code is used as a part of the fourth order predictor-corrector method. Below are the equations for the second order code block.}}

%acceleration equation
\[  \textit{\textbf {a}} = \sum_{i=1, i \neq j}^{N}  \frac{Gm_j} {\left| \bf {r_{ij}}^2 \right|}  \bf{\hat {r}_{ij}} \]  \begin{flushright} (1)  \end{flushright}

%position and velocity in second order
\[ {\bf r_1} = {\bf r_0} + {\bf v_0} dt + \frac{{\bf a_0}} {2} dt^2 \]
\begin{flushright} (2)  \end{flushright}

\[ {\bf v_1} = {\bf v_0} + \frac{({\bf a_0}+{\bf a_1})} {2} dt \]
\begin{flushright} (3)  \end{flushright}

where, 

$N$ - Number of bodies in the system 

$m_j$ - mass of the second object

${\bf {\hat {r}_{ij}}}$  - unit vector of the distance vector between two bodies

${\bf r_0, r_1}$ - current and future positions of the body

${\bf v_0, v_1}$ - current and future velocities of the body

${\bf a_0, a_1}$ - current and future accelerations of the body

%now talk about the bootstrapping

\end{document}