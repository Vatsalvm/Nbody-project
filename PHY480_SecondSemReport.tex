\documentclass{article}
\begin{document}

\title{N-Body simulations of star-star encounters}
\author{Vatsal Mandalia}
\maketitle

\begin{abstract}
{\normalsize{N-Body simulations are carried out on the Solar system using the predictor-corrector method. }}

\end{abstract}

\section{\textbf{Introduction}}
{\normalsize{Stars like our Sun are born in groups/clusters with other stars. All of them having a similar age, only a difference in their masses. Gravitational interactions in the clusters cause changes in the dynamical properties of the members. The evolution of this group of stars can be produced in N-Body simulations. In this project, N-Body simulations are carried out on the Solar System. Different methods were used which gave varying accuracies. 

\medskip

Star clusters are important targets in astronomical observations. A observation of the cluster represents a snapshot in time. Detailed spectroscopic observations can help in producing an H-R diagram for the cluster. This plot provides important information about the cluster like the different populations of stars, the age of the cluster etc. However, to look at how the cluster will vary with time under the influence of gravity, theoretical simulations represent the perfect choice. Here, by knowing the initial values of the of the dynamical properties of the members of the Solar system, we perform N-Body simulations.

\medskip

Two methods were used in this project over the two semesters. In the first semester, a second order Euler's method was employed. At the base of these methods lies two first order differential equations for acceleration and velocity. The Euler's method obtains the future position/velocity as an approximate solution of the differential equations. The errors occurring from this method go as $dt$$^{2}$. Higher accuracy is obtained with a smaller timestep, with a consequence of long computational time. Therefore, a higher order method is used to give a higher accuracy. A fourth order predictor-corrector method is applied in this semester for the N-Body simulations.}}

\section{Method}
\subsection{Bootstrapping}
{\normalsize{
The code for the fourth order predictor corrector method is divided into blocks for ease in error checking. Newton's law of gravity is the main theory behind these simulations of the Solar System. The second order code is used as a part of the fourth order predictor-corrector method. Below are the equations for the second order code block.}}

%acceleration equation
\[  \textit{\textbf {a}} = \sum_{i=1, i \neq j}^{N}  \frac{Gm_j} {\left| \bf {r_{ij}}^2 \right|}  \bf{\hat {r}_{ij}} \]  \begin{flushright} (1)  \end{flushright}

%position and velocity in second order
\[ {\bf r_1} = {\bf r_0} + {\bf v_0} dt + \frac{{\bf a_0}} {2} dt^2 \]
\begin{flushright} (2)  \end{flushright}

\[ {\bf v_1} = {\bf v_0} + \frac{({\bf a_0}+{\bf a_1})} {2} dt \]
\begin{flushright} (3)  \end{flushright}

\smallskip

where, 

\smallskip

$N$ - Number of bodies in the system 

$m_j$ - mass of the second object

${\bf {\hat {r}_{ij}}}$  - unit vector of the distance vector between two bodies

${\bf r_0, r_1}$ - current and future positions of the body

${\bf v_0, v_1}$ - current and future velocities of the body

${\bf a_0, a_1}$ - current and future accelerations of the body

\bigskip                         
{\normalsize{
For the next part of the project, this second order code will provide the past information. Eight paststeps are generated which are then bootstrapped into the predictor-corrector code. }}

\bigskip

\subsection{Predictor-Corrector method}
{\normalsize{
In comparison to the second order code used in the first semester, a fourth order predictor-corrector scheme allows for usage of small timesteps but not compensating in long computational time. In addition, higher accuracy is obtained from the fourth order predictor-corrector method.

A basic predictor-corrector method works on the basis of two steps:

\medskip

1. Predictor- This uses current value of the variable, ${\it y_l}$ in determining a predicted value, ${\it y_{l+1}}$ approximately.

\smallskip

2. Corrector- With the predicted value, ${\it y_{l+1}^P}$ and ${\it y_l}$, the corrected value at the ${\it (l+1)^{th}}$ step, ${\it y_{l+1}^C}$ can be calculated.

\medskip

In this project, the Adams-Bashforth-Moulton predictor-corrector (Conte and De Boor 1972) is applied. Being a multistep method, four starting points are required in the calculation of the future value. The predicted position, ${{\bf r}_{l+1}^P}$ and velocity, ${{\bf v}_{l+1}^P}$ is determined using the formulae below.

\[ {\bf r}_{l+1}^P = {\bf r}_l + \frac{{\it dt}} {24} (-9{\bf v}_{l-3} + 37{\bf v}_{l-2} - 59{\bf v}_{l-1} + 55{\bf v}_l) \]
\begin{flushright} (4) \end{flushright}

\[ {\bf v}_{l+1}^P = {\bf v}_l + \frac{{\it dt}} {24} (-9{\bf a}_{l-3} + 37{\bf a}_{l-2} - 59{\bf a}_{l-1} + 55{\bf a}_l) \]
\begin{flushright} (5) \end{flushright}

\smallskip

where,
\smallskip
${\bf v}_{l-3}, {\bf v}_{l-2}, {\bf v}_{l-1}, {\bf v}_l$ and ${\bf a}_{l-3}, {\bf a}_{l-2}, {\bf a}_{l-1}, {\bf a}_l$ are the positions and accelerations which resemble the past information.
\smallskip

Now, the predicted acceleration, ${\bf a}_{l+1}^P$ is determined by using ${\bf r}_{l+1}^P$ and ${\bf v}_{l+1}^P$ in an acceleration loop which uses equation (1). 
\medskip

With the ingredients obtained, the corrected position, ${\bf r}_{l+1}^C$ and velocity, ${\bf v}_{l+1}^C$ is evaluated from the formulae given below.
\smallskip

\[ {\bf r}_{l+1}^C = {\bf r}_l + \frac{{\it dt}} {24} ({\bf v}_{l-2} - 5{\bf v}_{l-1} + 19{\bf v}_l + 9{\bf v}_{l+1}^P) \]
\begin{flushright} (6) \end{flushright}

\[ {\bf v}_{l+1}^C = {\bf v}_l + \frac{{\it dt}} {24} ({\bf a}_{l-2} - 5{\bf a}_{l-1} + 19{\bf a}_l + 9{\bf a}_{l+1}^P) \]
\begin{flushright} (7) \end{flushright}

\smallskip

where,
\smallskip
${\bf v}_{l-2}, {\bf v}_{l-1}, {\bf v}_l, {\bf v}_{l+1}^P$ and ${\bf a}_{l-2}, {\bf a}_{l-1}, {\bf a}_l, {\bf a}_{l+1}^P$ become the past information given to the corrector.
\smallskip

As done after the predictor, the corrected acceleration, ${\bf a}_{l+1}^P$ is found from the corrected position and velocity. This predictor-corrector block is included in an infinite time loop which runs for a specific amount of time. 

Energy conservation check is then carried out to keep a track on the accuracy of this system. The fractional energy of the system is determined from the corrected position and velocity values.

\[ {\it KE} = \frac{1}{2} { \sum_{i=1}^N {\it m_i} {\it v_i^2}} \] 
\begin{flushright} (8) \end{flushright}

\[ {\it PE} = \sum_{i=1,i \neq j}^{N} \frac{\it Gm_i m_j}{\it r} \]
\begin{flushright} (9) \end{flushright}

where,
${\it v_i}$ - magnitude of the corrected velocity, ${\bf v}_{l+1}^C$
\[ {\it v_i^2 = v_{ix}^2 + v_{iy}^2 + v_{iz}^2} \]

${\it r}$ - magnitude of the corrected position, ${\bf r}_{l+1}^C$
\[ {\it r} = ({\it r_x}^2 + {\it r_y}^2 + {\it r_z}^2 )^{1/2}\]

The fractional energy is then given by,

\[ {\it \Delta E} = \frac{\it {E_{curr} - E_{init}}} {\it E_{init}} \]

where,
$\it E_{curr}$ - current total energy of the system at the particular iteration
\[ {\it E_{curr}} = {\it KE} + {\it PE} \]

$\it E_{init}$ - initial total energy of the system
\smallskip

Along with the determination of the fractional energy, the orbital distance of the planets was calculated. This gives an additional idea on the stability of the system. 
\[ {\it d_{sun}} = ({\it r_x^2} + {\it r_y^2})^{1/2} \]

where,
${\it r_x}$ and ${\it r_y}$ -  magnitude of the x,y component of the corrected position, ${\it r_{l+1}^C}$
}
\subsection{Adaptive timestep and error check}
{\normalsize{

This method has an advantage which allows for the timestep to vary to result in better accuracy. At each iteration, the predicted and corrected values are used in determining an error estimate. The estimated error is compared against a relative error, RelErr. A value of $5.0 * 10^{-6}$ is taken in the beginning. The formula below explains the calculation of this error, called as error factor, 'Errfact'

\[ \textrm{Errfact} = \frac{19}{270} \frac{\left| {\it y_{l+1}} - {\it p_{l+1}} \right|} {{\left| {\it y_{l+1}} \right|} + \textrm{Small}} \]
\smallskip

where,
${\it y_{l+1}}$ and ${\it p_{l+1}}$ are the corrected and predicted values for the x/y/z component of the position/velocity of a body respectively. 

Small - an offset used in the calculation with a fixed value of $10^{-5}$ 

The values for the position and velocity are considered in the error calculation. For each direction component of the position and velocity, the error factor is determined. The maximum value among the three directions is taken for each body. With these values obtained for all the bodies, a maximum of those is taken for the position and velocity. This ends with two final contributions to the error from the position and velocity.  Finally, the larger among the two errors is compared against the RelErr'. 

Two cases are considered in this comparison.
\smallskip

Case 1:  \[\textrm{Errfact} > \textrm{RelErr} \]   \[{\it dt} = 0.5{\it dt} \]

On halving the timestep, new interpolated steps, ${\bf v}_{l-{1/2}}$ and ${\bf v}_{l-{3/2}}$ are created as shown below,

\[ {\bf v}_{l-{1/2}} = \frac{-5{\bf v}_{l-4} + 28{\bf v}_{l-3} - 70{\bf v}_{l-2} + 140{\bf v}_{l-1} + 35{\bf v}_l} {128} \]

\[ {\bf v}_{l-{3/2}} = \frac{3{\bf v}_{l-4} - 20{\bf v}_{l-3} + 90{\bf v}_{l-2} + 60{\bf v}_{l-1} - 5{\bf v}_l} {128} \]

These formulae are similar for getting the interpolated steps for acceleration (${\bf a}_{l-{1/2}}$ and ${\bf a}_{l-{3/2}}$). 

where,
${\bf v}_{l-4}$, ${\bf v}_{l-3}$, ${\bf v}_{l-3}$, ${\bf v}_{l-2}$, ${\bf v}_{l-1}$, ${\bf v}_l$ are the four past steps required in this interpolation.

\medskip
Case 2: \[\textrm{Errfact} < \frac{\textrm{RelErr}} {100} \]    \[{\it dt} = 2{\it dt} \]

Doubling the timestep represents the easier task as there are no new past steps produced. So the first four past steps become, ${\bf v}_l$, ${\bf v}_{l-2}$, ${\bf v}_{l-4}$ and ${\bf v}_{l-8}$. }}

\medskip
\section{Results}
{\normalsize{
%NOW TALK ABOUT TESTING EACH BLOCK OF THE CODE: P-C CHECK, ADAPTIVE DT CHECK, ENERGYCONSERVATION AND STABILITY CHECK
%FINALLY TO INCLUDE THE FIGURES FOR THE TESTS, EN CONSER CHECK AND ORBITAL DIST. CHECK

\section{Conclusion}

\section{References}
Conte, S. D. and De Boor, C.,(1972). {\it Elementary numerical analysis: an algorithmic approach.} 2nd ed. New York: McGraw-Hill
%Goldstein, M.E. and Braun, W.H., (1973).{\it Advanced Methods for the Solution of Differential Equations.} Washington, D.C.:  

\section{Appendix}
\end{document}